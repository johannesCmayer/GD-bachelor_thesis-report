% !TEX root = BachelorBookletMain.tex

\chapter{Neural Network Systems}
\section{Model}
To create the neural network model used in this project the Kears funtional API is used. Keas is the official frontend to Tensorflow and simplifies the creation of neural networks. With the functional API predefined callable class instances---which represent layers---can be used to define model architectures. For this a class of a layer type is set up and called with the input to the layer. This then returns the output\footnote{The Keras classes don't execute statements immediately, but construct a computation graph for later execution.} of the layer. The outputs can now be feed into another layer. After the model has been described in this manner, Keras can receive optimization parameters e.g. the loss function and compile the executable computation graph.\todo{is this nessesary?}

The model created corresponds to the model described in section \ref{BackgroundGQN}, with the exception that no latent variables $z$ are used. As encoder and decoder dense networks are used.

\dl{cimg('image series of how the network evolved, show output of network')}


\subsection{Saving and loading of network}
To make it easier to identify and reuse models a simple versioning system has been developed. If no specific model is specified to be used a new one of the specified architecture is created. The model is given a unique name consistent of the training date, a randomly generated human personal name for quick identification, a random numerical id and a version number.

If request is received to load a specific model, an automatic search is performed for the newest model of that id, which is then loaded instead. The versioning of a model is set up properly based on the version of the model that is currently trained.

Keras is used to automatically save the model during training in intervals, preventing total data loss on a system failure.


\subsection{Data Preprocessing}
First the data is loaded from from disk. Then the data points are normalized. This means that we make all inputs of the data have a range between zero and one. This is required\footnote{There also exist alternative methods like batch normalization \cite{Ioffe2015-eh}.} to make the network train efficiently.
normalizing data,
concatenating,
pair up img with coord,
reconstructing dimentions


\section{Inter process communication}
UDP sending of player position to python

Motion JPEG over UDP socket


\section{Rendering}
In Unity the stream is decoded and rendered using custom Cg shaders. These shaders merges the network output stream with objects in the environment that are tagged to be visible in the combined render.
\dl{cimg('image with only network output')}
\dl{cimg('image unity + network output')}

Done by giving environment specific color, that is set to transparent, when unity render is layered over the network output.
\dl{cimg('img of unity environment where you see eveything has keyout color')}
Cull objects if hidden from camera pos in unity environment
