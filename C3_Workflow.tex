% !TEX root = BachelorBookletMain.tex

\chapter{Workflow}
Beside the iterative prototype workflow described below

As for implementing different systems needed for the prototype At a higher level I tried to follow the pattern of starting to researching a single component\footnote{such as the Kears functional API, UDP sockets, etc.} needed for the project. After enough information had been gathered I immediately tried to implement or improve the corresponding component.


\section{Prototypes}
Multiple protoypes explored: simple to complex


\subsection{Functional}
The first step in the project was to create a simple implementation of the rendering neural network. This first prototype should create a foundation that future prototypes could build upon. In this prototype the entire data generation- and preprocessing pipeline as well as the neural network model where implemented and evaluated. For this first prototype a small level with checkerboard textures and variable sky and wall colors between environments.\todo{make a distinction between environment and level everywhere in the document, where environment is a variation of a level}

\dl{cimg('img of simple checker beard room with one sky color')}
\dl{cimg('another sky color')}


\subsection{Walking sim}
\dl{cimg('picture without object')}
\dl{cimg('img of the morphing center object')}


\subsection{Top down}
The goal of this prototype is, to test how well the network could be used to hide information in a simple top town navigation environment.
The idea is, that the network

\dl{cimg('img of top down, clear sight on platform')}
\dl{cimg('top down, blurry when not on platform')}


\subsection{Object morphing}

\dl{cimg('img of the morphing center object')}
