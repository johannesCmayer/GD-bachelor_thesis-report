% !TEX root = BachelorBookletMain.tex

\chapter{Workflow}
In this project I fallowed an iterative prototype workflow creating a prototype to evaluate a specific ideas. As for implementing different systems needed for the prototype I followed the workflow of researching a single component\footnote{such as the Kears functional API, UDP sockets, etc.} and implementing it, before beginning further researching efforts.

What fallows is a description of the prototypes developed over the course of the project. The main prototype "maze game" was the main focus of the project after the direction of the project was defined using the results from previous prototypes as guidance. It is described in detail in \cref{MazeGameSystems}.


\subsection{Functional}
The first prototype was developed with the intension of establishing the core systems necessary for a neural network able to render a scene. The entire data generation- and preprocessing pipeline and the neural network model where implemented and evaluated. A small level with checkerboard textures and variable sky and wall colors between environments.\todo{make a distinction between environment and level everywhere in the document, where environment is a variation of a level}

\dl{cimg('img of simple checker beard room with one sky color')}
\dl{cimg('another sky color')}


\subsection{Walking sim}
\dl{cimg('picture without object')}
\dl{cimg('img of the morphing center object')}


\subsection{Top down}
The goal of this prototype is, to test how well the network could be used to hide information in a simple top town navigation environment.
The idea is, that the network

\dl{cimg('img of top down, clear sight on platform')}
\dl{cimg('top down, blurry when not on platform')}


\subsection{Object morphing}

\dl{cimg('img of the morphing center object')}
